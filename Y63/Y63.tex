\documentclass[a4paper, 14pt]{extarticle}

\usepackage{../latex/misc/preamble2}

\geometry{a4paper}

% Название дисциплины
\newcommand{\subject}{Физика} 

% Тип работы
% lab - для лабораторной работы 
% hw  - для домашней     работы
\newcommand{\task}{lab} 

% Номер работы
\newcommand{\taskNumber}{Я63} 

% Название работы
\newcommand{\taskNameOne}{} 
\newcommand{\taskNameTwo}{Радиоактивность.} 
\newcommand{\taskNameThree}{} 

% Имя студента
\newcommand{\studentName}{Очкин Н.В.}

% Имя преподававателя
\newcommand{\teacherName}{Применко А.Э.}

% Группа
\newcommand{\group}{ФН11-52Б}

% Вариант
\newcommand{\variant}{9}

\begin{document}

\graphicspath{ {../latex/images} } 
\normalsize

\newcommand{\printTask}{%
    \ifthenelse{\equal{\task}{lab}}{%
        лабораторной%
    }{%
        \ifthenelse{\equal{\task}{hw}}{%
            домашней%
        }{%
            Неизвестный тип задания%
        }%
    }%
}

\begin{titlepage}

    \begin{center}
        {\footnotesize \itshape Федеральное государственное бюджетное 
                       образовательное учреждение высшего образования}
    \end{center}

    \begin{minipage}[c]{0.1\textwidth}
        \includegraphics[width=1.1\textwidth]{iconBMSTU}
    \end{minipage}
    \hfill
    \begin{minipage}[c]{0.9\textwidth}
        \centering
        \itshape
        \bfseries
        \small
        \guillemotleft Московский государственный технический университет \\
        имени Н.Э. Баумана\guillemotright \\
        (национальный исследовательский университет) \\
        (МГТУ им. Н.Э. Баумана) 
    \end{minipage}

    \vspace{0.5cm}
    \noindent\rule{\textwidth}{2pt} \\

    \noindent\uline{\textbf{ФАКУЛЬТЕТ} ФУНДАМЕНТАЛЬНЫЕ НАУКИ} \\
    \vspace{-5pt} \\
    \noindent\uline{\textbf{КАФЕДРА} ВЫЧИСЛИТЕЛЬНАЯ МАТЕМАТИКА И МАТЕМАТИЧЕСКАЯ} \\
    \vspace{-5pt} \\
    \noindent\uline{ФИЗИКА (ФН11)} \\
    \vspace{-5pt} \\
    \noindent\uline{\textbf{НАПРАВЛЕНИЕ ПОДГОТОВКИ} МАТЕМАТИКА И КОМПЬЮТЕРНЫЕ} \\
    \vspace{-5pt} \\
    \noindent\uline{НАУКИ (02.03.01)} \\

    \begin{center}
        \bfseries
        \textsc{О т ч е т} \\[10pt]
        по \printTask {} работе \taskNumber
    \end{center}

    \vspace{10pt}

    \hspace{10pt} 
    \noindent \textbf{Название \printTask {} работы:} \par
    \vspace{5pt}
    \hspace{10pt} 
    \noindent \textbf{\uline{\taskNameOne}} \vspace{5pt} \\
    \null\hspace{31pt} 
    \textbf{\uline{\taskNameTwo}} \vspace{5pt} \\
    \null\hspace{31pt} 
    \textbf{\uline{\taskNameThree}}

    \vspace{10pt}

    \begin{center}
        \bfseries
        Вариант \textnumero {} \variant
    \end{center}

    \vspace{20pt}

    \hspace{10pt} 
    \noindent \textbf{Дисциплина:} \par
    \vspace{10pt}
    \hspace{10pt} 
    \noindent {\large \subject}

    \vspace{10pt}

    \begin{flushright}
        \renewcommand{\arraystretch}{3}
        \begin{tabular}{r r r}
            \multicolumn{1}{l}{Студент группы \uline{\group}} & 
            $\quad \underset{\text{(Подпись, дата)}}{\underline{\hspace{3cm}}} \quad$ & 
            \multicolumn{1}{c}{$\underset{\text{(И.О. Фамилия)}}{\uline{\textbf{\studentName}}}$} \\

            \multicolumn{1}{l}{Преподаватель} & 
            $\quad \underset{\text{(Подпись, дата)}}{\underline{\hspace{3cm}}} \quad$ & 
            \multicolumn{1}{c}{$\underset{\text{(И.О. Фамилия)}}{\uline{\textbf{\teacherName}}}$} \\
        \end{tabular}
    \end{flushright}

    \vfill

    \begin{center}
        \small
        Москва, 2024
    \end{center}
\end{titlepage}


\newgeometry{left=25mm, right=25mm, top=20mm, bottom=20mm}

\graphicspath{ {../latex/images/Y63} }

% Customize section, subsection, subsubsection and paragraph styles
\titleformat{\section}
  {\normalfont\large\bfseries}{\thesection}{1em}{}

\titleformat{\subsection}
  {\normalfont\normalsize\bfseries}{\thesubsection}{1em}{}

\titleformat{\subsubsection}
  {\normalfont\small\bfseries}{\thesubsubsection}{1em}{}

\titleformat{\paragraph}
  {\small\small\bfseries}{\theparagraph}{1em}{}

\setstretch{1}
\linespread{1.2}

\setlength{\parindent}{0pt}

\fontsize{14pt}{16pt}\selectfont

% --------------------------------------START--------------------------------------

\section*{Ход выполнения работы}\vspace{-20pt}\rule{\linewidth}{0.1mm}

\vfill

\begin{table}[h!]
    \centering
    \resizebox{\textwidth}{!}{ % Scale the table to the width of the text
    \setlength{\tabcolsep}{10pt} % Add horizontal space between columns
    \begin{tabular}{cccccc}
    \toprule
    $i$ & Фон & Препарат 0 & Препарат 1 & Препарат 2 & Препарат 3 \\
    \midrule
    1  & 6  & 54 & 9  & 20 & 22 \\
    2  & 10 & 51 & 13 & 23 & 18 \\
    3  & 7  & 51 & 10 & 17 & 19 \\
    4  & 8  & 63 & 7  & 22 & 26 \\
    5  & 7  & 57 & 10 & 21 & 23 \\
    6  & 8  & 55 & 9  & 20 & 24 \\
    7  & 6  & 66 & 9  & 18 & 20 \\
    8  & 9  & 49 & 8  & 17 & 20 \\
    9  & 9  & 67 & 13 & 24 & 23 \\
    10 & 5  & 59 & 8  & 16 & 24 \\
    11 & 9  & 52 & 10 & 20 & 22 \\
    12 & 8  & 43 & 8  & 18 & 25 \\
    13 & 5  & 56 & 9  & 19 & 24 \\
    14 & 10 & 57 & 13 & 14 & 26 \\
    15 & 6  & 56 & 7  & 23 & 24 \\ 
    \bottomrule
    \end{tabular}
    }
\end{table}

\vfill

\begin{alignat*}{2}
    \text{N}_\text{Ф} &= 226 & \hspace{100pt} \null & \\
    \text{N}_0 &= 1672 & \text{N}_0 - \text{N}_\text{Ф} &= 1446 \\ 
    \text{N}_1 &= 286 & \text{N}_1 - \text{N}_\text{Ф} &= 60 \\
    \text{N}_2 &= 584 & \text{N}_2 - \text{N}_\text{Ф} &= 358 \\
    \text{N}_3 &= 680 & \text{N}_3 - \text{N}_\text{Ф} &= 454 \\
\end{alignat*}

\vfill

\newpage

\begin{table}[h!]
    \centering
    \begin{tabular}{|l|l|}
    \hline
    Коэффициент регистрации & $f=0.04$ \\
    \hline
    Масса, г, KCl в препарате 3 & $m=25$ \\
    \hline
    Количество атомов $^{40} \mathrm{K}$ в препарате 3 & $N_{40}=2.368 \cdot 10^{19}$ \\
    \hline
    Скорость счета для калия, 1/с & $n_{p}= 1.681$ \\
    \hline
    Активность калия, Бк & $A=42.025$ \\
    \hline
    Период полураспада (в секундах и годах) & $T=3.905 \cdot 10^{17}$ \\
    \hline
    Погрешность измерения $T, \%$ & $\varepsilon= 3 \cdot 10^{10}$ \\
    \hline
    \end{tabular}
\end{table}

\begin{equation*}
    N_{40} = \delta N_A \cfrac{m}{M} \approx 2.368 \cdot 10^{19},
\end{equation*}
где \\
$N_A = 6.02 \cdot 10^{23}$ моль$^{-1}$ - число Авогадро, \\
$M = 75$ $\frac{\text{г}}{\text{моль}}$ - молярная масса соли, \\
$\delta = 1.18 \cdot 10^{-4}$ - доля радиоактивного изотопа. 

\begin{equation*}
    n_p = \cfrac{\text{N}_3 - \text{N}_\text{Ф}}{t} \approx 1.681,
\end{equation*}
где \\
$t$ = 270с - полное время 15 измерений.

\begin{equation*}
    A = \cfrac{n_p}{f} = 42.025
\end{equation*}

\begin{equation*}
    T = 0.693 \hspace{3pt} N_{40} / A \approx 3.905 \cdot 10^{17}
\end{equation*}
$T_{\text{табл}} = 1.3 \cdot 10^9$ лет.

\begin{equation*}
    \varepsilon = 100 \% (T - T_{\text{табл}}) / T_{\text{табл}} \approx 3 \cdot 10^{10} \hspace{3pt} \%
\end{equation*}

\begin{table}[h!]
    \centering
    \begin{tabular}{|l|l|l|}
    \hline
    Удельная активность, Бк/г, препарат 1 & $a=0.581$ & $\varepsilon=37.83$ \\
    \hline
    Содержание калия, \%, препарат 2 & $\Omega=2.174$ & $\varepsilon=37.83$ \\
    \hline
    Активность калия в теле человека, Бк & $A_{\text{ч}}=4200$ &  \\
    \hline
    Поглощенная энергия, Дж & $E=0.01056$ &  \\
    \hline
    Поглощенная доза, Гр & $D_{\mathrm{K}}=0.000075$ &  \\
    \hline
    Эквивалентная доза, Зв & $H_{\mathrm{K}}=0.0015$ &  \\
    \hline
    Доля годовой дозы & $H_{\mathrm{K}} / H=0.75$ &  \\
    \hline
    \end{tabular}
\end{table}

\begin{equation*}
    a = \cfrac{N - N_{\text{Ф}}}{N_0 - N_{\text{Ф}}} \cdot a_0 \approx 0.581,
\end{equation*}
где \\
$N = N_1 = 286$ - число импульсов от исследуемого препарата,\\
$N_{\text{Ф}} = 226$ - число импульсов фона,\\
$N_0 = 1672$ - число импульсов от эталонного препарата,\\
$a_0 = 14$ Бк / г - удельная активность соли KCl с испусканием $\beta$ - частиц.\\

\begin{equation*}
    \Omega = (52.4 \hspace{3pt} \%) \cfrac{N - N_{\text{Ф}}}{N_0 - N_{\text{Ф}}} = 2.174
\end{equation*}

\begin{equation*}
    \varepsilon = 100 \% \sqrt{\cfrac{N + N_{\text{Ф}}}{(N - N_{\text{Ф}})^2} + \cfrac{N_0 + N_{\text{Ф}}}{(N_0 - N_{\text{Ф}})^2}} \approx 37.83 \%
\end{equation*}

\begin{equation*}
    A_{\text{ч}} = m_{\text{ч}} a_0 = 4200,
\end{equation*}
где \\
$m_{\text{ч}} = 140$ г, \\
$a_0 = 30$ Бк / г

\begin{equation*}
    E = N \cdot E_1 \approx 0.01056, 
\end{equation*}
где \\
$N = A \cdot T$ - количество распадов в год, \\
$A = 4200$ - активность, \\
$T = 365 \cdot 24 \cdot 3600$ - количесвто секунд в году, \\
$E_1 = 0.5 \cdot 1.6 \cdot 10^{-13}$ Дж - энергия одного распада 

\begin{equation*}
    D = \cfrac{E}{m} \approx 75e-6
\end{equation*}

\begin{equation*}
    H = K D \approx 0.0015
\end{equation*}

\begin{equation*}
    \cfrac{H_{\mathrm{K}}}{H} = 0.75
\end{equation*}

\end{document}